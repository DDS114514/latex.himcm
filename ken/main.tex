\documentclass[11pt,a4paper]{ctexart} % 使用 ctexart 自动支持中文
\usepackage{graphicx}
\usepackage{geometry}
\usepackage{hyperref}
\geometry{left=2.5cm, right=2.5cm, top=2.5cm, bottom=2.5cm}

\title{城市垃圾回收数学建模项目}
\author{团队号:IMMC25954575}
\date{\today}

\begin{document}
\maketitle
\tableofcontents
\newpage

% ----------------------- 第一部分:介绍 -----------------------
\section{介绍}
本章节介绍项目背景、问题重述及基本假设。

\subsection{背景}
近年来,随着城市化进程加速,传统的垃圾填埋和焚烧处理方式面临着日益严峻的经济和环境挑战。垃圾回收作为一种可持续的解决方案,逐渐受到重视。本项目旨在通过数学建模的方法,探讨城市垃圾回收的策略优化问题。

\subsection{问题重述}
根据背景信息,本项目需要解决以下问题:
\begin{itemize}
  \item 分析各类垃圾的产生量、特性及回收成本与收益;
  \item 建立垃圾回收利润模型和环境污染评估模型;
  \item 通过模型求解确定最优的垃圾回收组合,并进行灵敏度分析;
\end{itemize}

\subsection{假设}
为便于建模,本文作出如下假设:
\begin{enumerate}
  \item 所有垃圾均为城市居民产生的生活垃圾;
  \item 垃圾回收由政府统一组织,工业垃圾不计入;
  \item 研究对象限定在中国某沿海发达城市;
\end{enumerate}

% ----------------------- 第二部分:任务1 - 数据收集 -----------------------
\section{任务1: 数据收集}
本章节描述项目中所需数据的来源与采集过程。依据相关文献,我们收集了包括垃圾产量、填埋与焚烧成本、回收价格等数据,为后续模型建立提供依据。例如,从文献\cite{Chen2018}和\cite{Ma2019}中获得了垃圾产量数据,文献\cite{Gao2018}提供了垃圾处理成本信息。

% ----------------------- 第三部分:任务2 - 模型建立 -----------------------
\section{任务2: 模型建立}
本章节主要介绍各模型的建立过程与理论基础,包括利润模型、环境污染评估模型和评分模型。

\subsection{结构}
模型的整体结构为多层次、多指标综合评价体系,根据不同垃圾的特性与回收成本,建立相应的数学表达式。

\subsection{废玻璃回收利润模型}
针对废玻璃回收的利润问题,我们建立了成本与收益的计算公式:
\[
c = b + p + h + r + w + o + m + l
\]
其中,\(b\) 表示原料收购成本,\(p\) 为预处理成本,\(h\) 为人工成本,\(r\) 为设备维护费用,\(w\) 为能耗成本,\(o\) 为其他费用,\(m\) 为机器成本,\(l\) 为土地成本。由此得出净利润计算方式。

\subsection{不同垃圾回收利润模型}
根据各类垃圾的回收价格及运输成本,我们计算出每日净利润:
\[
R = Q \times (P - T)
\]
其中,\(Q\) 为每日垃圾产量,\(P\) 为回收净利润,\(T\) 为运输成本。

\subsection{环境污染评估模型}
利用层次分析法(AHP)构建判断矩阵,对不同处理方式(填埋、焚烧)对环境的影响进行评估,并计算相应的环境影响指标。

\subsection{废品潜在污染性模型}
针对混合回收过程中可能出现的污染问题,建立了污染指数计算方法,通过垃圾之间的污染指数及其比例,量化污染风险。

\subsection{评分模型}
将各模型输出进行归一化处理,并按照预设权重综合评价各垃圾回收组合的总体效益:
\[
S = \sum_{i} R_i + E_j
\]
其中,\(R_i\) 表示各垃圾的利润分数,\(E_j\) 表示组合的总污染指数。

% ----------------------- 第四部分:任务3 - 灵敏度分析 -----------------------
\section{任务3: 灵敏度分析}
为验证模型的稳健性,本章节对模型进行了灵敏度分析。通过引入新的垃圾类别(如电池、小电器、衣物)并重新归一化数据,考察模型输出对参数变化的敏感程度。分析结果表明,模型能够较好地反映不同垃圾组合下的经济与环境效益变化。

% ----------------------- 第五部分:结论 -----------------------
\section{结论}
在综合模型计算和灵敏度分析的基础上,本文提出了一套针对城市垃圾回收的优化策略。主要结论包括:
\begin{itemize}
  \item 硬塑料、纸板和纸张的回收组合在利润与环境效益上表现较优;
  \item 模型为城市垃圾回收提供了定量依据,同时也揭示了数据采集与处理中的局限性;
  \item 未来研究可在数据细化及模型扩展方面进行深入探讨。
\end{itemize}

% ----------------------- 参考文献 -----------------------
\newpage
\begin{thebibliography}{99}
\bibitem{Chen2018} 陈倩倩, 杨栋, 黄颖, 等. 宁波市不同区分类垃圾组成与理化特性研究. \textit{环境科学学报}, 2018, 38(3): 1064--1070.
\bibitem{Ma2019} 马仕君, 周传斌, 杨光, 等. 城市生活垃圾填埋场的物质存量特征及其环境影响:以粤港澳大湾区为例. \textit{环境科学}, 2019, 40(12): 5593--5603.
\bibitem{Gao2018} 高健聪, 蒋沁芝, 聂正同, 等. 深圳市生活垃圾处理社会总成本的核算和预测. \textit{数学建模及其应用}, 2018, 7(2): 59.
\end{thebibliography}

\end{document}
