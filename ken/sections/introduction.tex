\section{介绍}
本章节介绍了项目的背景、问题重述以及基本假设。

\subsection{背景}
近年来,随着城市化进程加速,传统的垃圾填埋和焚烧处理方式面临着日益严峻的经济和环境挑战。垃圾回收作为一种可持续的解决方案,逐渐受到重视。本项目旨在通过数学建模的方法,探讨城市垃圾回收的策略优化问题。

\subsection{问题重述}
根据背景信息,本项目需要解决以下问题:
\begin{itemize}
  \item 分析各类垃圾的产生量、特性以及回收成本与收益;
  \item 建立垃圾回收利润模型和环境污染评估模型;
  \item 通过模型求解确定最优的垃圾回收组合,并进行灵敏度分析;
\end{itemize}

\subsection{假设}
为便于建模,本文在分析过程中作出如下假设:
\begin{enumerate}
  \item 所有垃圾均为城市居民产生的生活垃圾;
  \item 垃圾回收由政府部门统一组织,工业垃圾不计入;
  \item 研究对象限定在中国某沿海发达城市;
\end{enumerate}
