\section{任务2: 模型建立}
本章节主要介绍各个模型的建立过程与理论基础,包括利润模型、环境污染评估模型、以及评分模型。

\subsection{结构}
模型的整体结构设计为多层次、多指标综合评价体系,依据不同垃圾的特性与回收成本,建立相应的数学表达式。

\subsection{废玻璃回收利润模型}
针对废玻璃回收的利润问题,我们建立了成本与收益的计算公式:
\[
c = b + p + h + r + w + o + m + l
\]
其中,各项参数分别表示原料收购成本、预处理成本、人工成本等。由此得出净利润计算方式。

\subsection{不同垃圾回收利润模型}
通过对比各类垃圾的回收价格及运输成本,计算出每日净利润,具体公式为:
\[
R = Q \times (P - T)
\]
其中,\(Q\) 为每日垃圾产量,\(P\) 为回收净利润,\(T\) 为运输成本。

\subsection{环境污染评估模型}
利用层次分析法(AHP)构建判断矩阵,经过一致性检测后,计算不同处理方式(填埋、焚烧)对环境的影响。

\subsection{废品潜在污染性模型}
针对混合回收过程中可能出现的污染问题,建立了污染指数计算方法,通过垃圾之间的污染指数及其比例,量化污染风险。

\subsection{评分模型}
将各模型的输出进行归一化处理,并按照预设权重综合评价各垃圾回收组合的总体效益:
\[
S = \sum_{i} R_i + E_j
\]
其中,\(R_i\) 表示各垃圾的利润分数,\(E_j\) 表示组合的总污染指数。
